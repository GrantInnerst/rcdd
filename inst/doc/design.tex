
\documentclass{article}

\usepackage{indentfirst}
\usepackage{amsmath}

\RequirePackage{amsfonts}
\newcommand{\real}{\mathbb{R}}

\RequirePackage{amsmath}
\DeclareMathOperator{\lin}{lin}
\DeclareMathOperator{\aff}{aff}
\DeclareMathOperator{\pos}{pos}
\DeclareMathOperator{\con}{con}

\newcommand{\opand}{\mathbin{\rm and}}
\newcommand{\set}[1]{\{\, #1 \,\}}

\begin{document}

\title{Design of the RCDD Package}
\author{Charles J. Geyer}
\maketitle

\section{The Name of the Game}

We have called the package \verb@rcdd@ which stands
for ``C Double Description in R,'' our name being copied from
\verb@cddlib@, the library we call to do the computations.
This library was written by Komei Fukuda and is available at
\begin{verbatim}
    http://www.cs.mcgill.ca/~fukuda/soft/cdd_home/
\end{verbatim}
Our \verb@rcdd@ package for R makes available some
(by no means all) of the functionality
of the \verb@cddlib@ library.

The ``C'' is for either of the C or C++ computer languages.
This is bad terminology, making a mere implementation detail
part of the name, but we keep Fukuda's name.
And we make the terminology worse by tacking an ``R'' on the front
for another irrelevant implementation detail (but at least we're
consistent).

The two descriptions in question are the descriptions of a convex polyhedron
as either
\begin{itemize}
\item the intersection of a finite collection of closed half spaces or
\item the convex hull of of a finite collection of points and directions.
\end{itemize}

For those readers who are not familiar with the second description,
we give more detail.  We assume the notion of a point in $\real^d$
is familiar.  A \emph{direction} in $\real^d$ can be identified with
either a nonzero point $x$ or with
the ray $\{ \lambda x : \lambda \ge 0 \}$ generated by such a point.  
The \emph{convex hull} of a set of points $x_1$, $\ldots$, $x_k$ and
a set of directions represented as nonzero
points $x_{k + 1}$, $\ldots$, $x_m$ is the set of linear
combinations
$$
   x = \sum_{i = 1}^m \lambda_i x_i
$$
where the coefficients $\lambda_i$ satisfy
$$
   \lambda_i \ge 0, \qquad i = 1, \ldots, m
$$
and
$$
   \sum_{i = 1}^k \lambda_i = 1
$$
(note that only the $\lambda_i$ for points, not directions,
are in the latter sum).
The fact that these two descriptions characterize the same class of
convex sets (the \emph{polyhedral} convex sets) is Theorem~19.1
in Rockafellar (\emph{Convex Analysis}, Princeton University Press, 1970).
The points and directions are said to be \emph{generators} of the
convex polyhedron.  Those who like eponyms call this the
Minkowski-Weyl theorem
\begin{verbatim}
    http://www.ifor.math.ethz.ch/staff/fukuda/polyfaq/node14.html
\end{verbatim}

\section{Representations}

\subsection{The H-representation}

In the terminology of the \verb@cddlib@ documentation,
the two descriptions are called
the ``H-representation'' and the ``V-representation''
(``H'' for half space and ``V'' for vertex,
although, strictly speaking, generators are not always vertices).

For both efficiency and computational stability, the H-representation
allows not only closed half spaces but hyperplanes (which are, of course, the
intersection of two closed half spaces), or, what is equivalent,
the H-representation characterizes the convex polyhedron as the solution
set of a finite set of linear equalities and inequalities, that is,
the set of points $x$ satisfying
$$
   A_1 x \le b_1 \quad \text{and} \quad A_2 x = b_2
$$
where $A_1$ and $A_2$ are matrices and $b_1$ and $b_2$ are vectors
and the dimensions are such that these equations make sense.

In the representation used for our \verb@rcdd@ package
for R, these parts of the specification are combined into one big matrix
$$
   M = \begin{pmatrix} 0 & b_1 & - A_1 \\ 1 & b_2 & - A_2 \end{pmatrix}
$$
If the dimension of the space in which the polyhedron lives is $d$,
then $M$ has column dimension $d + 2$ and the first two columns are special.
The first column is an indicator vector, zero indicates an inequality
constraint and one an equality constraint.  The second column contains
the ``right hand side'' vectors $b_1$ and $b_2$.  Although we have given
an example in which all the inequality rows are on top of all the equality
rows, this is not required.  The rows can be in any order.

If \verb@m@ is such a matrix and we let
\begin{verbatim}
    l <- m[ , 1]
    b <- m[ , 2]
    v <- m[ , - c(1, 2)]
    a <- (- v)
\end{verbatim}
then the convex polyhedron described is the set of points \verb@x@ that
satisfy
\begin{verbatim}
    axb <- a %*% x - b
    all(axb <= 0)
    all(l * axb == 0)
\end{verbatim}

\subsection{The V-representation}

For both efficiency and computational stability, the V-representation
allows not only points and directions, but also lines and something I
don't know the name of (perhaps ``affine generators'').

In R a V-representation is matrix with the same column dimension as
the corresponding H-representation, and again the first two columns
are special,
but their interpretation is different.  Now the first two columns
are both indicators (zero or one valued).  The rest of each row
represents a point.

The convex polyhedron described is the set of linear combinations of
these points such that the coefficients are (1) nonnegative if
column one is zero and (2) sum to one where the sum runs over
rows having a one in column two.

If \verb@m@ is such an object and we define \verb@v@, \verb@b@, and
\verb@l@ as in the preceding section (\verb@l@ is column one, \verb@b@ is
column two, and \verb@v@ is the rest), then the polyhedron in question
is the set of points of the form
\begin{verbatim}
    y <- t(lambda) %*% v
\end{verbatim}
where \verb@lambda@ satisfies the constraints
\begin{verbatim}
    all(lambda * (1 - l) >= 0)
    sum(b * lambda) == max(b)
\end{verbatim}

\subsection{Fukuda's Representations}

Readers interested in comparing with Fukuda's documentation should be
aware that \verb@cddlib@ uses different but mathematically equivalent
representations.
If our representation is a matrix \verb@m@, then Fukuda's representation
consists of a matrix, which is our \verb@m[ , -1]@ and a vector
(which he calls the \emph{linearity}), which is our
\verb@seq(1, nrow(m))[m[ , 1] == 1]@
(that is the vector of indices of the rows having a one in our column one).

\section{Converting Between Representations}

The R function \verb@scdd@ converts H-representations to V-representations
and vice versa.
The result is a list that
always contains a component \verb@output@ which is the
computed representation and may contain a component \verb@input@ which is
the input representation (depending on an argument \verb@keepinput@,
about which see below).

Other options involve auxiliary computations, any of the arguments
\begin{verbatim}
    adjacency = TRUE
    incidence = TRUE
    inputadjacency = TRUE
    inputincidence = TRUE
\end{verbatim}
(the defaults are \verb@FALSE@)
produce additional results, which are components of the list returned
by \verb@scdd@ having the same name as the argument (\verb@adjacency@
and so forth).  Each is a ragged array: \verb@adjacency[[i]][j]@
(note the brackets) says that the \verb@i@-th and \verb@j@-th rows
of the \verb@output@ are ``adjacent'', and so forth.  See
\begin{verbatim}
    http://www.cs.mcgill.ca/~fukuda/soft/cddman/node4.html
\end{verbatim}
for more about these.

The result contains a component \verb@input@ if
\begin{verbatim}
    keepinput = "TRUE"
\end{verbatim}
or if
\begin{verbatim}
    keepinput = "maybe"
\end{verbatim}
and the input is involved in an adjacency or incidence list
(the default is \verb@"maybe"@).

The last option involves the computation itself.
The \verb@roworder@ option specifies the order in which the rows of $M$
are processed which can have a considerable effect on the running time
of the algorithm and, when using normal floating point arithmetic
(see Section~\ref{sec:gmp} below), on the numerical results of the
algorithm or even on success or failure of the algorithm.  This argument
is a finite choice
\begin{verbatim}
rowoder = c("lexmin", "lexmax", "minindex", "maxindex",
    "mincutoff", "maxcutoff", "mixcutoff", "randomrow")
\end{verbatim}
and \verb@match.arg@ is used for the argument matching, so
(1) the argument may be abbreviated and (2) the default is \verb@"lexmin"@
if no argument is specified.
\begin{verbatim}
http://www.cs.mcgill.ca/~fukuda/soft/cddman/node4.html
http://www.cs.mcgill.ca/~fukuda/soft/cddlibman/node6.html
\end{verbatim}
contain some discussion of which to use.  The main bit of advice seems
to be that \verb@roworder = "maxcut"@ might be useful when an input
H-representation contains many redundant inequalities or an input
V-representation contains many interior points.

\section{Using GMP Rational Arithmetic} \label{sec:gmp}

The \verb@cddlib@ code can also use the GMP (GNU Multiple Precision) Library
to compute results using exact arithmetic with unlimited precision rational
numbers.

In order to do this, the input problem must be in this form.  Thus we
need a way to specify rational numbers.  We specify them as character
objects of the following form: an optional minus sign followed by an
integer (the \emph{numerator}), followed by a slash, followed by another
integer (the \emph{denominator}).  If the denominator is one, both it
and the slash may be omitted.  The string contains no whitespace.

\pagebreak[1]
All of
\begin{verbatim}
    1/3
    -5/7
    2
    123456789012345567890123456789/33
\end{verbatim}
are valid.  Note that the last is not exactly representable as an
ordinary floating point number (for that matter neither are $1 / 3$
and $- 5 / 7$).  The point of the long example is to point out that
integers of any size are allowed.  The numerator and denominator do
not have to be representable as ordinary computer integers.

We have two functions \verb@d2q@ and \verb@q2d@ that convert from
standard floating point (\verb@storage.mode@ \verb@"double"@ in R)
to rational and vice versa.  One can also construct rationals
from numerators and denominators using the \verb@z2q@ function.


\end{document}

